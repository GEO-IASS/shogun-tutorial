% Permission is granted to copy, distribute and/or modify this document
% under the terms of the GNU Free Documentation License, Version 1.3
% or any later version published by the Free Software Foundation;
% with no Invariant Sections, no Front-Cover Texts, and no Back-Cover Texts.
% A copy of the license is included in the section entitled "GNU
% Free Documentation License".
%
% Written (C) 2012 Heiko Strathmann

\chapter{Statistical Testing}
This chapter describes SHOGUN's framework for statistical hypothesis testing. We begin by giving a brief outline of the problem setting in section \ref{sec:hypothesis_testing_into}. Then, we describe methods for two-sample testing for independence testing in section.

Methods for two-sample testing currently consist of tests based on the \emph{Maximum Mean Discrepancy}, section \ref{sec:mmd_into}. There are two types of tests available, a quadratic time test, which is described in section \ref{sec:mmd_quadratic}; and a linear time test, which is described in section \ref{sec:mmd_linear}. Both come in various flavours.

Independence testing is currently based in the \emph{Hilbert Schmidt Independence Criterion}, which is described in section \ref{sec:independence_testing_into} along with a test using it, which is described in section \ref{sec:hsic_test}

\section{Statistical Hypothesis Testing}
\label{sec:hypothesis_testing_into}

To set the context, we here briefly describe statistical hypothesis testing.

\section{Two-Sample-Testing with the Maximum Mean Discrepancy}
\label{sec:mmd_into}

\subsection{Quadratic Time MMD Statistic}
\label{sec:mmd_quadratic}

\subsubsection{Spectrum Approximation}
\subsubsection{Gamma Approximation}
\subsubsection{Bootstrapping}

\subsection{Linear Time MMD Statistic}
\label{sec:mmd_linear}
\subsubsection{Gaussian Approximation}
\subsubsection{Bootstrapping}

\section{Independence Testing with the Hilbert-Schmidt Independence Criterion}
\label{sec:independence_testing_into}

\subsection{HSIC Statistic}
\label{sec:hsic_test}